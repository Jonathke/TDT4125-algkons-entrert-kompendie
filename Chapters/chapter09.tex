\section{Preflyt}
De forrige algoritmene brukte på et eller annet vis alle augmenting paths for å finne maksimal flyt. Men ved kun å bruke slike paths kan vi få ganske dårlig kjøretid ved å konstruere spesifikke eksempler (der vi trenger mange augmenting paths, som alle kun øker flyten med 1). I denne seksjonen skal vi se en litt annen maks-flyt variant, som unngår dette problemet, og som er en av de raskere maks-flyt algoritmene idag, både teoretisk og praktisk.

\subsection{Push-Relabel}
Den algoritmen baserer seg på distance labels (så vidt nevnt i slutten av forrige seksjon), og såkalt preflyt. En preflyt har samme krav om capacity constraint samt skew symmetry fra flyt, men har et løsere krav enn flow conservation: nemlig at for alle $i \in V, i \neq s$ så har vi $\sum\nolimits_{k:(k,i)\in A}f(k,i) \geq 0$ (sagt med ord at det kommer mer flyt inn enn det går ut av hver node, husk vi har skew symmetry). Vi kaller denne mengden \emph{excess at $i$ for a preflow $f$}, og skriver $e_f(i)=\sum\nolimits_{k:(k,i)\in A}f(k,i)$. Legg også merke til at en flyt er og en gyldig preflyt.

Videre ser vi på distance labels $d(i), \forall i \in V$. Disse representerer på et vi en nedre grense for distansen til sluket. En intuisjon som blir nyttig etterhvert er og at de representerer "høyden" til node i. Vi har en gyldig distance labeling om følgende krav er oppfylt:

\begin{itemize}
  \item $d(s) = n$
  \item $d(t) = 0$
  \item $d(i) \leq d(j) + 1, \forall (i,j) \in A_f \text{(residualgrafen)}$
\end{itemize}

Videre har vi at for en preflyt $f$ og en valid distance labeling $d$, finnes ingen augmenting paths i $G_f$. For å se dette, anta at det finnes en s-t path $P$ i $G_f$ (altså en augmenting path). P inneholder maks $n$ noder, og dermed maks $n-1$ kanter. Da får vi $d(s)=d(t) + |P| \leq 0 + n-1 = n-1$. Men $d(s)=n$, så dette er en kontradiksjon. 

Dette gir oss alt vi trenger for å outline push-relabel algoritmen. Vi starter med en valid distance labeling og en preflyt. Deretter forsøker vi gradvis å gjøre om preflyten til en flyt, mens vi holder distance labelingen valid. Dette resulterer da i en maks flyt på grunn av resultatet over.

Litt mer spesifikt ønsker vi å dytte så mye \emph{excess at i} mot sluket, over de kantene som er på den \emph{tilsynelatende} korteste veien. Med det menes at vi kun dytter flyt over en kant $(i,j)$, dersom $d(i) = d(j) + 1$ (om dette og $u_f(i,j) > 0$ holder, kaller vi kanten admissable). Når vi dytter dytter vi alltid mest mulig, dvs. $min(e_f(i),u_f(i,j))$. Om vi ikke har noen admissable kanter for en node $i$ med $e_f(i) > 0$, da tolker vi det som at labelen er feil, og gjør en relabel. Mer spesifikt øker vi $d(i)$ til $d(i) = min_{j:(i,j) \in A_f}(d(j)+1)$, dvs. den laveste $j$ som $i$ har residual-flyt til. Merk at en slik $j$ alltid finnes, pga. skew symmetry kan vi alltids dytte flyt tilbake igjen. Her er $d(i)$ som høyde en god intuisjon, der flyt alltid går nedover.

Om vi nå initialiserer push-relabel algoritmen vår som $d(s)=n, d(i)=0, i \neq s$, og setter $f(s,k) = u(s,k) \forall k$, så ser vi at algoritmen holder en gyldig preflyt og en valid distance labeling hele veien. Dette er ikke vanskelig å se, se evt. boka/forelesning. Det er og lett å se at når algoritmen stopper (da $e_f(i) = 0, \forall i \neq s$) har vi en flyt, og som argumentert for tidligere er den maksimal.

\subsection{Kjøretid}
Analysen av kjøretiden baserer seg hovedsaklig på to resultater. Den ene sier at for enhver preflyt $f$, for en node $i, e_f(i) > 0$, så finnes det alltid en path fra i til s i $G_f$. Det andre resultatet er at for en node $i$ har vi $d(i) \leq 2n-1$. Dette fordi at for at vi skal relable så må $e_f(i) > 0$, og da finnes det en path $P$ til $s$. Dette gir at $d(i) \leq d(s) + |P| \leq 2n - 1$.

Se nå på antall relabel operasjoner. På starten har vi $d(i) = 0, \forall i\neq s$. Videre er som vist alltid $d(i) \leq 2n - 1$. Hver relable øker $d(i)$ med minst 1. Siden vi aldri endrer på $s,t$ har vi $n-2$ noder som kan relables. Da får vi et maks antall relable operasjoner som $(n-2)(2n-1) = \mathcal{O}(n^2)$. Analysen av antall pushes er hakket mer tricky, men fortsatt greit. Vi deler inn i saturerende ($u(i,j)$ blir dyttet) og ikke-saturerende (mindre enn $u(i,j)$ blir dyttet) pushes. Samme som for i shortest path har vi en øvre grense på $\mathcal{O}(mn)$ ganger å saturere kanter. For å se det, se at for at en path $(i,j)$ skal satureres 2 ganger, må $d(i)$ øke med minst to, men vi har og $d(i) \leq 2n-1$. Analysen av antall ikke-saturerende pushes er smart. Vi konstruerer en funksjon $\phi = \sum\nolimits_{i active}d(i)$. Den slutter som 0, og er aldri negativ. Det sentrale er at et ikke-saturerende push senker $\phi$ med minst 1. For å se dette, se at selv om pushet gjør $j$ aktiv hadde vi jo $d(i) = d(j)+1$ (ikke-saturerende betyr jo at vi dytter $e_f(i)$, så $i$ blir jo nødvendigvis inaktiv). Dermed blir en øvre grense for ikke-saturerende dytt gitt av hvor mye $\phi$ kan øke med under algoritmen. Ved relabeling kan vi øke $\phi$ med $\mathcal{O}(n^2)$, siden vi har $n-2$ noder som kan økes med $2n-1$. Videre vil også saturerende push øke $\phi$, med opp til $2n-1$ ($i$ blir ikke nødvendigvis inaktiv, mens $j$ kan bli aktiv). Som vi har sett kan vi ha $\mathcal{O}(mn)$ slike dytt og vi får en øvre grense for saturerende dytt på $\mathcal{O}(n^2m)$ Dermed blir også en øvre grense for ikke saturerende dytt $\mathcal{O}(n^2m)$. Alle operasjoner tar $\mathcal{O}(1)$ tid, så total kjøretid på push-relabel algoritmen blir dermed også $\mathcal{O}(n^2m)$.

\subsection{Forbedringer}
Vi har en forbedrert versjon av push-relabel kalt highest label push-relabel, som forbedrer kjøretiden til $\mathcal{O}(n^2\sqrt{m})$, der vi alltid dytter fra den aktive noden $i$ med høyest $d(i)$. Kjøretidsanalysen her er hakket mer teknisk, men går frem stort sett på likt vis som i forrige avsnitt.

Vi har også små forbedringer, som ikke forbedrer den teoretiske kjøretiden, men som fremdeles kan bety mye for den praktiske kjøretiden. Det første er at vi tidlig kan finne $|f|$ uten å vite $f$ ($f$ betegner her den maksimale flyten). Den baserer seg på å redefinere aktive noder til kun å være aktive hvis de også har $d(i) < n$. Dette gir $|f|$ på grunn av et resultat som sier at "om vi terminererer push-relabel algoritmen når $e_f(i) > 0$ impliserer $d(i) \geq n, i \neq t$, da er settet $S$ av noder som ikke kan nå $t$ i $A_f$ et min s-t cut". Se detaljer i boka/forelesning. Et annet raskt triks da er "gap relabeling", som sier at om det opstår et gap i distance labelsene, typ en $k < max_i(d(i))$, så setter vi $d(j) = n, \forall j:k < d(j) < n$. Et siste triks som eksperimentelt har vist seg å være lurt å gjøre av og til (ca. hver $n$ relabel) er å regne faktisk distanse til $t$ for alle noder, og relable $d(i)$ til den korrekte distansen.
